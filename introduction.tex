This paper describes the hardware and software aspects of the RoboFEI@Home team, designed to compete in the RoboCup 2016 @Home League.

The group has a long tradition in Robotic Soccer and the first time we took part in a competition was in 1998, when Prof. Reinaldo Bianchi was a member of the group that developed the FutePOLI Team, which competed in the First Brazilian Micro Robot Soccer Cup, held in São Paulo, Brazil. Since then, Bianchi started the development of soccer playing robots at the Centro Universitário FEI, competing in the Very Small Size category and after, in the Very Small Size league. In 2004, during the First Brazilian Competition on Intelligent Robots (a competition were the IEEE Robot Competition was held together with the First Brazilian RoboCup), our team became Brazilian Champion in the IEEE Very Small category. In the same competition, our first team developed for the RoboCup Small Size League became vice-champion In 2006, Very Small Size team became Champion again, and our first 2D RoboCup Simulation team was the Brazilian champion.

The institution RoboCup Small Size League team, called RoboFEI, won for the first time the Brazilian Robocup in 2010, and is currently the Brazilian champion, winning the championship 4 times in a row (2010, 2011, 2012 and 2013). This team takes part in the RoboCup World Competition since 2009, and the best result we had was in 2012, when we stayed among the 8 top teams. After developing robotic soccer players for the last 17 years, we developed a team to compete in the RoboCup Humanoid League. The development of this team started in 2012, with students designing and building a humanoid robot from scratch. Last year, the RoboFEI-HT team competed our first RoboCup World Competition, held in João Pessoa, PB, Brazil, with 4 humanoid robots: two Newton (developed in the institution, being the pieces of one of them, made in a 3D printer) and two humanoids robots based on Darwin-OP. At RoboCup 2014, the team stayed among the 16 top teams and in the same year, the team competed the Latin American Robotics Competition (LARC 2014) and became champion in the LARC RoboCup Humanoid Kid Size league.

Now, the institution has establish important positions on small size and humanoid league, we decide to drive our researches into new paths for colaborating on social and service robots. Last year, we applied for Latin American @Home competition and we won the 3rd place.