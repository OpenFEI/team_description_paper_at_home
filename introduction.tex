\section{Introduction}

This paper describes the hardware and software aspects of the RoboFEI@Home team, designed to compete in the RoboCup 2016 @Home League.

The first steps towards the first group of research in robots at Centro Universitário da FEI was made by Prof. Reinaldo Bianchi in 2002 with the first groups of students in RoboCup Simulation 2D. Next year, with the experience of Prof. Reinaldo Bianchi who have participated in the soccer robot group in USP since 1998, together with Prof. Flavio Tonidandel, developed the first robots for Very Small Size Category of the IEEE Robotic Competition. In 2004, during the First Brazilian Competition on Intelligent Robots (a competition in which the IEEE Robot Competition was held together with the First Brazilian RoboCup), our team became Brazilian Champion in the IEEE Very Small category. In the same competition, our first team developed for the RoboCup Small Size League became vice-champion In 2006, Very Small Size team became Champion again, and our first 2D RoboCup Simulation team was the Brazilian champion.

The institution RoboCup Small Size League team, called RoboFEI, won for the first time the Brazilian Robocup in 2010, and is currently the Brazilian champion, winning the championship 6 times in a row (2010, 2011, 2012, 2013, 2014 and 2015). This team takes part in the RoboCup World Competition since 2009, and the best result we had was in 2012, when we stayed among the top 8 teams. After developing robotic soccer players for the last 17 years, we developed a team to compete in the RoboCup Humanoid League. The development of this team started in 2012, with students designing and building a humanoid robot from scratch. Last year, the RoboFEI-HT team competed our first RoboCup World Competition, held in João Pessoa, PB, Brazil, with 4 humanoid robots: two Newton (developed in-house, being the pieces of one of them, made in a 3D printer) and two humanoids robots based on Darwin-OP. At RoboCup 2014, the team stayed among the 16 top teams and in the same year, the team competed the Latin American Robotics Competition (LARC 2014) and became champion in the LARC RoboCup Humanoid Kid Size league.

Now that the institution has establish important positions on small size and humanoid league, we decide to drive our research into new paths for collaborating on social and service robots. The robot design, particularly the human factor concerns, are a key aspect of human-robot interaction (HRI). Research in HRI attractions from similar research in human-computer interaction (HCI) but features a number of significant differences related to the robot’s physical real-world personification. This team was born in Usability Engineering Laboratory, now defined as Human Robot Interaction and Intelligent Interfaces Group (HR3iGroup). This research group has abundant knowledge in HCI and is focused on integrating with HRI. The robot’s physical personification, simplicity or complexity of design, form and level of anthropomorphism, human behavior and robot interaction feedback, robotic reactions and humanized relationships, are some of the key research areas being explored.

Last year, we applied for and achieved 3rd place at the Latin American @Home competition. We focus on research applied in Human-Robot Interaction, Social and Service Robots, Behavior Analysis, People Modeling and Profiling, and all questions for improving people and robots coexistence.
