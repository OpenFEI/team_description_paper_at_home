
%%%%%%%%%%%%%%%%%%%%%%% file typeinst.tex %%%%%%%%%%%%%%%%%%%%%%%%%
%
% This is the LaTeX source for the TDPTemplate using
% the LaTeX document class 'llncs.cls' Springer LNAI format
% used in the RoboCup Symposium submissions.
% http://www.springer.com/computer/lncs?SGWID=0-164-6-793341-0
%
% It may be used as a template for your own TDP - copy it
% to a new file with a new name and use it as the basis
% for your Team Description Paper
%
% NB: the document class 'llncs' has its own and detailed documentation, see
% ftp://ftp.springer.de/data/pubftp/pub/tex/latex/llncs/latex2e/llncsdoc.pdf
%
%%%%%%%%%%%%%%%%%%%%%%%%%%%%%%%%%%%%%%%%%%%%%%%%%%%%%%%%%%%%%%%%%%%

\documentclass[runningheads,a4paper]{llncs}
\usepackage{amssymb}
\setcounter{tocdepth}{3}
\usepackage{graphicx}
\usepackage{amssymb}
\usepackage[utf8]{inputenc}
\usepackage{url}
\usepackage{float}
\usepackage{amsmath}
\usepackage{graphicx}
\usepackage{wrapfig}
\usepackage{subfigure}

% *** MORE GRAHPICS ***
\usepackage[usenames,dvipsnames]{color}     

% *** BIBLIOGRAPHY PACKAGES ***
%\usepackage{natbib}     

%%%%%%%%%%%%%%%%%%%%%%%%%%%%%%%%%%%%%%%%%%%%%%%%%%%%%%%%%%%%%%%%%%%
\usepackage{booktabs}           % For tables (toprule, midrule, bottomrule)
%%%%%%%%%%%%%%%%%%%%%%%%%%%%%%%%%%%%%%%%%%%%%%%%%%%%%%%%%%%%%%%%%%%

%%%%%%%%%%%%%%%%%%%%%%%%%%%%%%%%%%%%%%%%%%%%%%%%%%%%%%%%%%%%%%%%%%%
% *** PATHS ***
% \makeatletter
% \def\input@path{{figures/}		
% 			   }
% \makeatother

% \graphicspath{{figures/}
% 			 }
%%%%%%%%%%%%%%%%%%%%%%%%%%%%%%%%%%%%%%%%%%%%%%%%%%%%%%%%%%%%%%%%%%%

%%%%%%%%%%%%%%%%%%%%%%%%%%%%%%%%%%%%%%%%%%%%%%%%%%%%%%%%%%%%%%%%%%%
\newcommand{\eg}{\emph{e.g.}}						% Exemplum gratia
\newcommand{\ie}{\emph{i.e.}}						% Id est
%%%%%%%%%%%%%%%%%%%%%%%%%%%%%%%%%%%%%%%%%%%%%%%%%%%%%%%%%%%%%%%%%%%

\begin{document}

\title{RoboFEI@Home 2016 Team Description Paper}
\author{Andrey~A.~Masiero, Douglas~de~Rizzo~Meneghetti, Leonardo~Contador, Lucas~Vasconcelos, Flavio~Tonidandel and Plinio~T.~Aquino~Junior}
\institute{Centro Universitário FEI,
\newline Av. Humberto Alencar de Castelo Branco, 3972, 09850-901, SBC, SP, Brazil\\
\texttt{http://www.fei.edu.br/robofei, amasiero@fei.edu.br}}
\authorrunning{A.A.~Masiero et al.}

\maketitle


%%%%%%%%%%%%%%%%%%%%%%%%%%%%%%%%%%%%%%%%%%%%%%%%%%%%%%%%%%%%%%%%%%%%%%%%%%%%%%%%%%%%

\begin{abstract}
This paper presents the description of the RoboFEI@Home team as it stands for the Robocup 2016 in Leipzig, Germany. It describes all mechanical, electrical and software modules, designed for RoboFEI's robot, which is called Judith, a robot based on Peoplebot platform. Judith was built to attempt Human-Robot Interaction (HRI) on social scenarios, which is the main research of our team. A framework for behavior analysis and adaptation have been contructed in the last year to be implement on Judith and many other robots.
\end{abstract}

\keywords{Robocup@Home, RoboFEI, Peoplebot, Autonomous Robot}
%%%%%%%%%%%%%%%%%%%%%%%%%%%%%%%%%%%%%%%%%%%%%%%%%%%%%%%%%%%%%%%%%%%%%%%%%%%%%%%%%%%%

\section{Introduction}
\section{Introduction}

This paper describes the hardware and software aspects of the RoboFEI@Home team, designed to compete in the RoboCup 2016 @Home League.

The first steps towards the first group of research in robots at Centro Universitário da FEI was made by Prof. Reinaldo Bianchi in 2002 with the first groups of students in RoboCup Simulation 2D. Next year, with the experience of Prof. Reinaldo Bianchi who have participated in the soccer robot group in USP since 1998, together with Prof. Flavio Tonidandel, developed the first robots for Very Small Size Category of the IEEE Robotic Competition. In 2004, during the First Brazilian Competition on Intelligent Robots (a competition in which the IEEE Robot Competition was held together with the First Brazilian RoboCup), our team became Brazilian Champion in the IEEE Very Small category. In the same competition, our first team developed for the RoboCup Small Size League became vice-champion In 2006, Very Small Size team became Champion again, and our first 2D RoboCup Simulation team was the Brazilian champion.

The institution RoboCup Small Size League team, called RoboFEI, won for the first time the Brazilian Robocup in 2010, and is currently the Brazilian champion, winning the championship 4 times in a row (2010, 2011, 2012 and 2013). This team takes part in the RoboCup World Competition since 2009, and the best result we had was in 2012, when we stayed among the top 8 teams. After developing robotic soccer players for the last 17 years, we developed a team to compete in the RoboCup Humanoid League. The development of this team started in 2012, with students designing and building a humanoid robot from scratch. Last year, the RoboFEI-HT team competed our first RoboCup World Competition, held in João Pessoa, PB, Brazil, with 4 humanoid robots: two Newton (developed in-house, being the pieces of one of them, made in a 3D printer) and two humanoids robots based on Darwin-OP. At RoboCup 2014, the team stayed among the 16 top teams and in the same year, the team competed the Latin American Robotics Competition (LARC 2014) and became champion in the LARC RoboCup Humanoid Kid Size league.

Now that the institution has establish important positions on small size and humanoid league, we decide to drive our research into new paths for collaborating on social and service robots. The robot design, particularly the human factor concerns, are a key aspect of HRI. Research in these areas attractions from similar research in human-computer interaction (HCI) but features a number of significant differences related to the robot’s physical real-world personification. This team was born in Usability Engineering Laboratory, now defined as Human Robot Interaction and Intelligent Interfaces Group (HR3iGroup). This research group has abundant knowledge in HCI and is focused on integrating with HRI. The robot’s physical personification, simplicity or complexity of design, form and level of anthropomorphism, human behavior and robot interaction feedback, robotic reactions and humanized relationships, are some of the key research areas being explored.

Last year, we applied for and won 3rd place at the Latin American @Home competition. We focus on research applied in Human-Robot Interaction, Social and Service Robots, Behavior Analysis, People Modeling and Profiling, and all questions for improving people and robots coexistence.


\section{Hardware Design}
\section{Hardware Design}

Our team has one robot, called Judith (see Fig.~\ref{fig:judith}). Judith is a robot based on the Peoplebot robot~\cite{peoplebot:2001}, using a KUKA youBot arm~\cite{youbot:2016} as its main manipulator. As shown in Fig.~\ref{fig:judith}, Judith is composed of: 1 Microsoft Kinect for Xbox 360; 2 sonar arrays; 2 infrared sensors; 1 iPad 2; 1 Yoga HT-320A microphone; 1 touch sensor arrays; 2 speakers; 1 emergency button; and 1 Hokuyo URG-04LX-UG01 laser sensor.

\begin{figure}[ht!]
    \centering
    \includegraphics[height = 7.2cm]{figures/judith_info.png}
    \caption{RoboFEI's Judith}
    \label{fig:judith}
\end{figure}

Working with two different pre-manufactured robots present some challenges. First of all, each one have electric source with different voltage. KUKA youBot arm needs 24v to work while Peoplebot works with 12v. To put Peoplebot and KUKA youBot arm together, we need to build a circuit for connecting both battery kits together. This circuit is important to keep all parts of the robots safe in case of electric damage on overcharge. It also helps to make both robots work at the same time.

Peoplebot has a vertical grip which was removed for the KUKA youBot arm to be attached. To attach the new arm, we built a support to keep it on the right height to manipulate objects on top of an ordinary sized table. KUKA youBot arm weighs 16.53 pounds so a wooden support was made to hang up with this weight.

With both the mechanical and electrical projects setup, we need to connect both robots on main computer. In order to connect Peoplebot to our laptop we used a serial RS232 to USB converter. The manipulator is able to transmit messages through an EtherCAT cable, which is connected to the computer's network port.

The robot's main computer has been an ASUS Ultrabook 14'' Touch-Screen Laptop, with an Intel Core i5, 4GB Memory, 500GB Hard Drive. It has only 3 USB ports supporting all devices, so an USB hub is used to enable more ports and increase the number of interfaced devices. As the iPad needs a paid annual license for software development, a change is being made towards Android technology so as to allow us to develop an interactive face for Judith.

The platform used by Peoplebot makes transportation a challenge, so we have worked on a modular platform using a combination of 3D print technology and aluminum parts. We want to accomplish this project until Robocup 2016 in Germany.


\section{Software Design}
%This section present all details about RoboFEI@Home software. A stable version of the software is available on GitHub\footnote{https://github.com/OpenFEI/rfh\_judite}. Follow sections are divided as: (I) Software Architecture of Judith; (II) Face and People Recognition; (III) Object Detection; (IV) Speech Recognition and Synthesize; (V) Navigation; (VI) Object Manipulation.

\subsection{Software Architecture}\label{architecture}
All Judith software is developed using Ubuntu Linux 14.04 LTS~\cite{sobell:2014} and ROS Indigo Igloo~\cite{ros:2015}. The initial architecture of Judith is composed by three layers. Each layer is responsible for one part of the code. First layer receives all data from sensors, extract the features and publish it for AI and Controller nodes (Second Layer). AI and Controller nodes take the features and run algorithms for computer vision, location, planning, pattern recognition, among others. In the end, second layer send to third layer what is the velocity of the motors, position of manipulator or face express by robot. Third layer controls all the actuators drivers and actions. A graphic view of the architecture is presented on Fig.~\ref{fig:architecture}.

\begin{figure}[ht!]
    \centering
    \includegraphics[width = \textwidth]{figures/architecture.png}
    \caption{Software Architecture of Judith}
    \label{fig:architecture}
\end{figure}

Sensors and Actuators layers has been developed using C++~\cite{stroustrup:1986}, due to hardware drivers communication, except by judith\_face node. Face's node is developed in Java~\cite{joy:2000} language to create an application on Android~\cite{android:2016} Platform. AI/Controllers layer is developed using Python~\cite{vanrossum:2010} as programming language to all algorithms. All decisions made for the best performance of our robot during task execution and also for training new members to the team.

\subsection{Face and People Recognition}\label{face-people-recognition}
We understand when you recognize a person and treat it by the name, you make the interaction more comfortable for her/him. In that way, we develop some nodes that can detect and recognize a person by their face. To make it possible, we acquire an image using a camera (Webcam Logitech C920 or Microsoft Kinect X360) and convert the frame into a OpenCV~\cite{bradski:2000} Image Matrix. This information is a input of the LBP algorithm which has been widely used for face recognition due to its computational performance~\cite{ahonen:2006,yang:2007,shan:2012,ylioinas:2012,samadi:2013}. LBP algorithm may also help on identify gender, age and facial expression which can be useful for making an adaptive behavior of the robot during interaction.

Beyond facial recognition this set of nodes is responsible for following a single person, without external interference during process. To perform this task, we decide to combine three techniques. First is face detection followed by extraction of the color from person t-shirt~\cite{pulli:2012,laganiere:2011,baggio:2012}. With the t-shirt color defined, we perform a color segmentation using OpenCV~\cite{kang:2008,oliveira:2009,culjak:2012}. The last technique is to extract a skeleton position using Nite/OpenNI library for Microsoft Kinect~\cite{openni:2011}. With this information robot can follow the direction of a certain person and also to identify the distance between them.

\subsection{Object Detection}\label{object-detection}

\subsection{Speech Recognition and Synthesize}\label{speech}
Speech is one of the most important way to interact with a human. Due to that, we use CMU Pocketsphinx~\cite{huggins:2006} to make Judith's speech recognition and also synthesize a female voice on our robot.

\subsection{Navigation Stack}\label{navigation}
To execute tasks on environment the robot needs to know where it is and how to arrive into another point of the place. The first step is create an environment map. For performing this task, we use GMapping and SLAM to create the map with laser Hokuyo URG-04LX-UG01 helps. We save the map bag using a node from ROS called map\_server.

After that, we run the bag saved and execute the algorithm \emph{Adaptive Monte Carlo Localization} (AMCL)~\cite{fox:1999} with data from the laser make the robot capable to locate its position on the map. With position known, we can give the robot a point to go on the map. Next step is to use an algorithm A* to make the route between current position and desire position. After that, for each movement of the robot, it executes an \emph{Dynamic Window Approach}(DWA) algorithm for planning the current position movements.

\subsection{Object Manipulation}\label{manipulation}
For manipulating an object, we first use object detection node (see sec.~\ref{object-detection}) to know its current position. Then we execute a few movements preview determined to put the gripper on robot view. After that, we use PCL algorithm~\cite{aldoma:2012} to segment the gripper and get object distance. At the end we execute planning to move KUKA youBot arm joints for getting closer and grab the object.


\section{Conclusions}
%\section{Conclusions}

We have described all development of our robot Judith. We believe that all of these tasks can help on home, health and office services. To improve the acceptance of these robots, we have been working into a behavior analysis framework developed using ROS. With this framework, we intend to:

\begin{itemize}
    \item Identifying people's emotion;
    \item Creating user behavior profile;
    \item Improving approach to start a interaction.
\end{itemize}

With this TDP, we hope to participate on Robocup 2016 in Germany and try to advance during the competition and learn with all teams, as we did to win third place on XIV Latin American Robotic Competition in 2015. Hope to see you all there.


\section{Bibliography}
\bibliographystyle{splncs03}
\bibliography{refs}
\end{document} 